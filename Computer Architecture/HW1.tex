\documentclass{article}
\usepackage{amsmath}
\begin{document}
\paragraph{HW1 Zhijia Chen}\text{ }\\

\noindent\textbf{1.1}\\
\indent \textbf{a.}\\
\begin{align*}
    \text{Die area} &= 2\ cm^2\\
    \text{Yield} &= \frac{1}{{(1+\text{Defects per unit area}\times \text{Die area})}^N}\\
    &= \frac{1}{{(1+0.04\times 2)}^{14}}\\
    &=0.341
\end{align*}

\indent \textbf{b.}\\
\indent Because Phoenix has smaller manufacturing size than BlueDragon and thus its manufacturing is more difficult.

\noindent\textbf{1.4}\\
\indent \textbf{a.}\\
\indent For core running at full power:\\
\begin{align*}
    Power_{dynamic}&=\text{number of cores}\times \text{full power}\\
    &=4\times \text{0.5 W}\\
    &=\text{2 W}\\
    Energy_{dynamic}&=Power_{dynamic}\times T\\
    &=2T
\end{align*}
Where T is the time required for the phone to finish the task when running at full power.\\
\indent For core running 1/8 of the time:
\indent The workload, capacity and voltage are not changed, so the required dynamic energy remains the same as running at full power, i.e., 
\begin{equation*}
    Energy_{dynamic}=2T
\end{equation*}
\indent The average dynamic power would reduced to 1/8
\begin{align*}
    Power_{dynamic}&=2/8\\
    &=\text{0.25 W}
\end{align*}

\indent \textbf{b.}\\
\indent Since the frequency and the voltage are both reduced to 1/8 the entire time, $Energy_{dynamic}\propto\text{Capacitive load}\times\text{Voltage}^2$ and $Power_{dynamic}\propto\text{Capacitive load}\times\text{Voltage}^2\times\text{Frequency}$\\
\begin{align*}
    Energy_{dynamic}&=\left(\frac{1}{8}\right)^2\times 2T\\
    &=\frac{1}{64}\times 2T\\
    &=\frac{1}{32}T\\
    Power_{dynamic}&=\left(\frac{1}{8}\right)^2\times\frac{1}{8}\times \text{2 W}\\
    &=\frac{1}{512}\times\text{2 W}\\
    &=\frac{1}{256}\text{W}
\end{align*}
\indent \textbf{c.}\\
\indent If voltage reduced to 1/2 and frequency reduced to 1/8:
\begin{align*}
    Energy_{dynamic}&=\left(\frac{1}{2}\right)^2\times 2T\\
    &=\frac{1}{4}\times 2T\\
    &=\frac{1}{2}T\\
    Power_{dynamic}&=\left(\frac{1}{2}\right)^2\times\frac{1}{8}\times \text{2 W}\\
    &=\frac{1}{32}\times\text{2 W}\\
    &=\frac{1}{16}\text{W}
\end{align*}
\indent \textbf{d.}\\
No idea...\\

\noindent\textbf{1.7}\\
\indent \textbf{a.} $2^5=32$\\
\indent \textbf{b.} The clock rate would be $\text{5 MHz}\times 1.4^(2025-1978)\approx\text{37 THz}$\\
\indent \textbf{c.} In the year of 2017, the chip has the clock rate at 4200 MHz, and the current rate of increase is 2\%, thus the projected performance in 2025 is $\text{4200 MHz}\times 1.02^(2025-2017)\approx\text{4920 MHz}$.\\
\indent \textbf{d.} The Moore's law has ended, the number of transistors on a chip has reached its limit, also the heat dissipation has also becomes a problem, hampering the further increasing of clock rate.\\
\indent \textbf{e. } By current DRAM growth rate, the capacity doubles in 4 years, thus the growth rate is $2^{0.25}\approx 1.189$.\\

\noindent \textbf{1.9}\\
For the following question, I suppose that the server being turned off or put in "barely live" state or reduced voltage and frequency are running at 60\% of capacity and consuming 90\% of the maximum power.\\
\indent \textbf{a.} The saving would be 0.9$\times$ 0.6$\times$ maximum operate power, that is, 54\% of the maximum operate power.\\
\indent \textbf{b.}
The saving would be 0.9$\times$ (0.6-0.2)$\times$ maximum operate power, that is, 36\% of the maximum operate power.\\
\indent \textbf{c.} The power saving would be $1-{(1-0.2)}^2\times(1-0.4)=0.616$, that's 61.6\% of the current running power, or (0.616$\times$ 0.9) of the maximum power, i.e., 55.44\% of the maximum power.\\
\indent \textbf{d.}\\
The saving would be (54\%/2+36\%/2) of the maximum power, i.e., 45\% of the maximum power.\\

\noindent \textbf{1.10.}\\
\indent \textbf{a.}\\
\indent That means the MTTF=$\frac{10^9}{100}=10^7$.\\
\indent \textbf{b.}\\
\indent availability=$\frac{\text{MTTF}}{(\text{MTTF+MTTR})}=\frac{10^7}{(10^7+24)}\approx$0.999998\\
\indent \textbf{c.}\\
\indent Assume that the lifetimes are exponentially distributed and the failures are independent:
\begin{align*}
    \text{Failure rate}_{\text{super computer}}&=1000\times \frac{1}{\text{MTTF}_\text{processor}}\\
    &=\frac{1000}{10^7}\\
    &=10^{-4}\\
    \text{MTTF}_\text{super computer}&=\frac{1}{\text{Failure rate}_{\text{super computer}}}\\
    &=\text{10000 hours}
\end{align*}

\noindent \textbf{1.16.}\\
\indent \textbf{a.}\\
\indent speedup = $\frac{1}{0.2+\frac{0.8}{N}}$\\
\indent \textbf{b.}\\
\indent speedup = $\frac{1}{0.2+\frac{0.8}{8}+8\times0.005}$ = 2.941\\
\indent \textbf{c.}\\
\indent speedup = $\frac{1}{0.2+\frac{0.8}{8}+3\times0.005}$ = 3.
175\\
\indent \textbf{d.}\\
\indent speedup = $\frac{1}{0.2+\frac{0.8}{N}+log(N)\times0.005}$\\
\indent \textbf{e.}\\
speedup function:$f(N) = \frac{1}{1-P+\frac{P}{N}+log(N)\times0.005}$, make $\frac{df(N)}{dN}=0$, i.e.,
\begin{equation*}
    \frac{d\left(\frac{1}{1-P+\frac{P}{N}+log(N)\times0.005}\right)}{dN}=0
\end{equation*}

\noindent\textbf{A3}\\
Instruction mix for gobmk and mcf:\\
\textbf{Loads:} 28\%\\
\textbf{Stores:} 11.5\%\\
\textbf{Branches:} 19\%\\
\textbf{Jumps:} 1.5\%\\
\textbf{ALU operations:} 39.5\%\\
\textbf{others:} 0.5\%\\
\begin{align*}
    \text{effective CPI}&=\sum{\text{Instruction category frequency}\times\text{Clock cycles for category}}\\
    &=0.28\times 3.5+0.115\times 2.8+0.19\times (0.6\times 4+(1-0.6)\times 2)\\
    &\ \ \ \ \ \ \ \ \ \ \ \ \ \ \ \ \ +0.015\times 2.4+0.395\times 1 + 0.005\times 3\\
    &=2.356
\end{align*}

\noindent\textbf{A9}\\
\indent \textbf{a.}\\
\indent Yes. For 3 two-address instructions, we can use (00, 01, 10) of the first two bits to represent the 3 instructions and the remaining 12 bits to hold the two addresses. Then the first two bits of all other instructions must be (11). For the 63 one-address instructions, we start it with (11) and use the following 6 bits to present the 63 instructions excluding (11000000), and the last 6 bits to hold the address. And then we start the 45 zero-address instructions with (11000000) and use the remaining 6 bits to represent the 45 instructions.\\
\indent \textbf{b.}\\
\indent Impossible.
\end{document}
