\documentclass{article}
\usepackage{booktabs}
\usepackage{amsmath}
\usepackage{graphicx}
\usepackage{caption}
\title{HW5}
\author{Zhijiia Chen}
\begin{document}
\maketitle

\paragraph{3.1} The base performance is $11+3+4+10+3+2+2+1+1=37$ cycles per loop iteration.

\paragraph{3.14 a.} As shown in the following table, unscheduled instruction takes 19 clock cycles per loop iteration, while scheduled instruction takes 13 clock cycles. So the scheduled instruction is 31.58\% faster, and the clock should be 31.58\% faster for the unscheduled instruction to match the performance of scheduled instruction.
\begin{figure}[!ht]
\makebox[1 \textwidth][c]{
\resizebox{0.9 \textwidth}{!}{
\begin{tabular}{c|l|l} % <-- Alignments: 1st column left, 2nd middle and 3rd right, with vertical lines in between
    \toprule
    \textbf{clock cycle}&\textbf{scheduled instruction}&\textbf{unscheduled instruction}\\
    \hline
    1&addi x4,x1,\#800&addi x4,x1,\#800\\
    \hline
    2&fld F2,0(x1)&fld F2,0(x1)\\
    \hline
    3&stall&fld F6,0(x2)\\
    \hline
    4&fmul.d F4,F2,F0&fmul.d F4,F2,F0\\
    \hline
    5&fld F2,0(x1)&addi x1,x1,\#8\\
    \hline
    6&stall&addi x2,x2,\#8\\
    \hline
    7&stall&sltu x3,x1,x4\\
    \hline
    8&stall&stall\\
    \hline
    9&stall&stall\\
    \hline
    10&fadd.d F6,F4,F6&fadd.d F6,F4,F6\\
    \hline
    11&stall&bnez x3,foo\\
    \hline
    12&stall&stall\\
    \hline
    13&stall&stall\\
    \hline
    14&fsd F6,0(X2)&fsd F6,-8(X2)\\
    \hline
    15&addi x1,x1,\#8&fld F2,0(x1)\\
    \hline
    16&addi x2,x2,\#8&fld F6,0(x2)\\
    \hline
    17&sltu x3,x1,x4&fmul.d F4,F2,F0\\
    \hline
    18&stall x3,x1,x4&addi x1,x1,\#8\\
    \hline
    19&bnez x3,foo&sltu x3,x1,x4\\
    \hline
    20&stall&stall\\
    \hline
    21&fld F2,0(x1)&stall\\
    \hline
    22&stall&fadd.d F6,F4,F6\\
    \hline
    23&fmul.d F4,F2,F0&bnez x3,foo\\
    \hline
    24&...&...\\
    \bottomrule
\end{tabular}
}
}
\end{figure}

\paragraph{3.14 b.} As shown in the following table, the loops need to be unrolled three times. The three loops take 19 clock cycles, so each iteration now takes 6.33 cycles.

\begin{figure}[!ht]
    \centering
    %\makebox[1 \textwidth][c]{
    %\resizebox{0.9 \textwidth}{!}{
    \begin{tabular}{c|l} % <-- Alignments: 1st column left, 2nd middle and 3rd right, with vertical lines in between
        \toprule
        \textbf{clock cycle}&\textbf{scheduled instruction}\\
        \hline
        1&add    x4,x1,\#800\\
        \hline
        2&fld    F2,0(x1)\\
        \hline
        3&fld    F6,0(x2)\\
        \hline
        4&fmul.d    F4,F2,F0\\
        \hline
        5&fld    F2,8(x1)\\
        \hline
        6&fld    F8,8(x2)\\
        \hline
        7&fmul.d    F10,F2,F0\\
        \hline
        8&fld    F2,16(x1)\\
        \hline
        9&fld    F12,16(x2)\\
        \hline
        10&fmul.d    F14,F2,F0\\
        \hline
        11&fadd.d    F6,F4,F6\\
        \hline
        12&addi    x1,x1,\#24\\
        \hline
        13&fadd.d    F8,F10,F8\\
        \hline
        14&addi    x2,x2,\#24\\
        \hline
        15&sltu    x3,x1,x4\\
        \hline
        16&fadd.d    F12,F14,F12\\
        \hline
        17&fsd    F6,-24(X2)\\
        \hline
        18&fsd    F8,-16(X2)\\
        \hline
        19&bnez    x3,foo\\
        \hline
        20&fsd    F12,-8(X2)\\
        \hline
        21&...\\
        \bottomrule
    \end{tabular}
    %}
    %}
    \end{figure}
\newpage

\paragraph{3.15 a} See the following table.

\begin{figure}[!ht]
    \makebox[1 \textwidth][c]{
    \resizebox{1 \textwidth}{!}{
    \begin{tabular}{c|l|c|c|c|c|p{2.5cm}} % <-- Alignments: 1st column left, 2nd middle and 3rd right, with vertical lines in between
        \toprule
        \textbf{Iteration}&\textbf{Instruction}&\textbf{Issues}&\textbf{Executes}&\textbf{Memory access}&\textbf{Write CDB}&\textbf{Comment}\\
        \hline
        1&fld F2,0(x1)&1&2&2&3&\\
        \hline
        1&fmul.d F4,F2,F0&2&4& &19&wait for F2\\
        \hline
        1&fld F2,0(x1)&3&4&4&5& \\
        \hline
        1&fadd.d F6,F4,F6&4&20& &30&wait for F4\\
        \hline
        1&fsd F6,0(X2)&5&31&31&&wait for F6\\
        \hline
        1&addi x1,x1,\#8&6&7& &8& \\
        \hline
        1&addi x2,x2,\#8&7&8& &9& \\
        \hline
        1&sltu x3,x1,x4&8&9& &10& \\
        \hline
        1&bnez x3,foo&9&11& &&wait for x3\\
        \hline
        2&fld F2,0(x1)&10&12&12&13&wait for bnez\\
        \hline
        2&fmul.d F4,F2,F0&11&19& &34&wait for F2 and \newline FP multiplier\\
        \hline
        2&fld F2,0(x1)&12&13&&14& \\
        \hline
        2&fadd.d F6,F4,F6&13&35& &45&wait for F4\\
        \hline
        2&fsd F6,0(X2)&14&46&46&&wait for F6\\
        \hline
        2&addi x1,x1,\#8&15&16& &17& \\
        \hline
        2&addi x2,x2,\#8&16&17& &18& \\
        \hline
        2&sltu x3,x1,x4&17&18& &19& \\
        \hline
        2&bnez x3,foo&18&20& &&wait for x3\\
        \hline
        3&fld F2,0(x1)&19&21&21&22&wait for bnez\\
        \hline
        3&fmul.d F4,F2,F0&20&34& &49&wait for F2 and \newline FP multiplier\\
        \hline
        3&fld F2,0(x1)&21&22&22&23& \\
        \hline
        3&fadd.d F6,F4,F6&22&50& &60&wait for F4\\
        \hline
        3&fsd F6,0(X2)&23&61&61&&wait for F6\\
        \hline
        3&addi x1,x1,\#8&24&25& &26& \\
        \hline
        3&addi x2,x2,\#8&25&26& &27& \\
        \hline
        3&sltu x3,x1,x4&26&27& &28& \\
        \hline
        3&bnez x3,foo&27&29& &&wait for x3\\
        \bottomrule
    \end{tabular}
    }
    }
    \end{figure}
    \newpage

\paragraph{3.17} 

For the correlating predictor, as shown in the following table, there are 3 misprediction, so the misprediction rate is 33.33\%.
\begin{figure}[!ht]
    \makebox[1 \textwidth][c]{
    \resizebox{0.9 \textwidth}{!}{
    \begin{tabular}{c|c|c|c|l} % <-- Alignments: 1st column left, 2nd middle and 3rd right, with vertical lines in between
        \toprule
        \textbf{Branch PC mod 4}&\textbf{Entry}&\textbf{Prediction}&\textbf{Outcome}&\textbf{Update}\\
        \hline
        2&4&T&T&None\\
        \hline
        3&6&NT&NT&change prediction to "NT"\\
        \hline
        1&2&NT&NT&None\\
        \hline
        3&7&NT&NT&None\\
        \hline
        1&3&T&NT&change prediction to "T with one misprediction"\\
        \hline
        2&4&T&T&None\\
        \hline
        1&3&T&NT&change prediction to "NT"\\
        \hline
        2&4&T&T&None\\
        \hline
        3&7&NT&T&change prediction to "NT with one misprediction"\\
        \bottomrule
    \end{tabular}
    }
    }
    \end{figure}

For the local predictor, as shown in the following table, there are 4 misprediction, so the misprediction rate is 44.44\%.
\begin{figure}[!ht]
    \makebox[1 \textwidth][c]{
    \resizebox{0.9 \textwidth}{!}{
    \begin{tabular}{c|c|c|c|l} % <-- Alignments: 1st column left, 2nd middle and 3rd right, with vertical lines in between
        \toprule
        \textbf{Branch PC mod 4}&\textbf{Entry}&\textbf{Prediction}&\textbf{Outcome}&\textbf{Update}\\
        \hline
        0&0&T&T&change prediction to "T"\\
        \hline
        1&4&T&NT&change prediction to "T with one misprediction"\\
        \hline
        1&5&T&NT&change prediction to "NT"\\
        \hline
        1&7&NT&NT&None\\
        \hline
        1&7&NT&NT&None\\
        \hline
        0&0&T&T&None\\
        \hline
        1&7&NT&NT&None\\
        \hline
        0&0&T&T&None\\
        \hline
        1&7&NT&T&change prediction to "NT with one misprediction"\\
        \bottomrule
    \end{tabular}
    }
    }
    \end{figure}
\end{document}