\documentclass{article}
\usepackage{booktabs}
\usepackage{amsmath}
\usepackage{graphicx}
\usepackage{caption}
\title{HW5}
\author{Zhijiia Chen}
\begin{document}
\maketitle

\paragraph{3.1} The base performance is $11+3+4+10+3+2+2+1+1=37$ cycles per loop iteration.

\paragraph{3.14} 

\begin{table}[ht!]
\begin{center}
\begin{tabular}{l|c|c|c} % <-- Alignments: 1st column left, 2nd middle and 3rd right, with vertical lines in between
    \toprule
    \textbf{iteration}&\textbf{instruction}&\textbf{issued at}&\textbf{completes at}\\
    \hline
    1&fld F2,0(x1)&1&3\\
    \hline
    1&fmul.d F4,F2,F0&2&IF&ID&EX&WB&\\
    \hline
    i+2&&&Stall&IF&ID&EX&WB&\\
    \hline
    i+3&&&&&IF&ID&EX&WB\\
    \bottomrule
    \end{tabular}
\end{center}
\end{table}

\subparagraph{a.} What is the average memory access time of the current cache (in cycles) versus the way-predicted cache?

Suppose a cache hit time is 2 cycles for current cache. And for the way-predicated cashe, the hit time is 1 cycle if it prediction success and 3 cycles otherwise.

Average memroy access time for current cache: 2 + 0.030 $\times$ 20 = 2.6 cycles.

Average memroy access time for way-predicated cache: $0.97\times(0.8\times1+0.2\times3)+0.03\times(3+20)=2.048$ cycles.

\subparagraph{b.} If all other components could operate with the faster way-predicted cache cycle time (including the main memory), what would be the impact on performance from using the way-predicted cache?

$\frac{2.6}{2.048}=1.27$, so the performance could be 27\% faster.



\paragraph{3.15} Consider the usage of critical word first and early restart on L2 cache misses. Assume a 1 MB L2 cache with 64-byte blocks and a refill path that is 16 bytes wide. Assume that the L2 can be written with 16 bytes every 4 processor cycles, the time to receive the first 16 byte block from the memory controller is 120 cycles, each additional 16 byte block from main memory requires 16 cycles, and data can be bypassed directly into the read port of the L2 cache. Ignore any cycles to transfer the miss request to the L2 cache and the requested data to the L1 cache.

\subparagraph{a.} How many cycles would it take to service an L2 cache miss with and without critical word first and early restart?

Each block contains 4 words. For early restart, suppose each word has the same possibility to be the expected one.

Without critical word first and early restart, it requires 120 cycles for the first word, and 16 cycles for each of the following 3 words, thus $120+3\times16=168$ cycles.

With critical word first, it requires 120 cycles for the expected word.

With early restart, it requires 120 cycles for the first word, and the expected cycles to wait for the remaining ones. So the expected cycles = $120 + 0.25\times(0 + 16 + 32 + 48) = 148$ cycles.

\paragraph{3.17} You are designing a write buffer between a write-through L1 cache and a write-back L2 cache. The L2 cache write data bus is 16 B wide and can perform a write to an independent cache address every four processor cycles.

\end{document}