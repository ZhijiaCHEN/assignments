\documentclass{article}
\usepackage{graphicx}
\usepackage{subcaption}
\usepackage{geometry}
\usepackage{tikz}
\usepackage{amsmath}
\usepackage{cleveref}
\usepackage{float}
\usepackage[useregional]{datetime2}
\def\checkmark{\tikz\fill[scale=0.4](0,.35) -- (.25,0) -- (1,.7) -- (.25,.15) -- cycle;}
\usepackage[font=small,skip=0pt]{caption}
\geometry{legalpaper, margin=1in}
\title{CIS 5525 Project 1}
\author{Zhijia Chen}
\date{\today}

\begin{document}

\begin{titlepage}
    \maketitle
\end{titlepage}

\textbf{Task 0.}
\vspace{\baselineskip}
In this task, we use the following loss function:

\begin{align*}
    loss = \sum_{x\in \chi}{\left(\sum_{k=1}^{K}\|x-u_k\|^2\frac{e^{-\alpha\|x-u_k\|^2}}{\sum_{k'=1}^{K}e^{-\alpha\|x-u_{k'}\|^2}}\right)}
\end{align*}

The number of target clusters $K$ is set to 3. The data is normalized so that its mean is 0 and standard deviation is 1. We apply SGD to optimize the loss function. Figure ... shows the clustering results and the loss curve function over epochs with different $\alpha$ values. The upper two sub-figures shows the results when $\alpha=1$ and the bottom two are for $\alpha=3$. We find that the loss function is likely to converge faster with a bigger $\alpha$ value, but 
\begin{figure}[h!]
    \centering
    \begin{subfigure}{.33\textwidth}
      \centering
      \includegraphics[width=.9\linewidth]{task0-cluster-alpha-1.pdf}
      \caption{Clustering result for $\alpha$=1.}
      \label{fig:cluster3}
    \end{subfigure}%
    \begin{subfigure}{.66\textwidth}
      \centering
      \includegraphics[width=.9\linewidth]{task0-loss-alpha-1.pdf}
      \caption{Loss function curve for $\alpha$=1.}
      \label{fig:loss3}
    \end{subfigure}
    \centering
    \begin{subfigure}{.33\textwidth}
      \centering
      \includegraphics[width=.9\linewidth]{task0-cluster-alpha-3.pdf}
      \caption{Clustering result for $\alpha$=3.}
      \label{fig:cluster3}
    \end{subfigure}%
    \begin{subfigure}{.66\textwidth}
      \centering
      \includegraphics[width=.9\linewidth]{task0-loss-alpha-3.pdf}
      \caption{Loss function curve for $\alpha$=3.}
      \label{fig:loss3}
    \end{subfigure}
    \caption{Clustering result and loss curve function with different $\alpha$.}
    \label{task0}
\end{figure}

\textbf{Task 1.}
\vspace{\baselineskip}

\end{document}

