\documentclass{article}
\usepackage{graphicx}
\usepackage{subcaption}
\usepackage{cleveref}
\usepackage{geometry}
\geometry{legalpaper, margin=1in}
\begin{document}
\noindent\textbf{1.} Compose an Insertion Sort program to sort N=10,000 randomly generated integers.\\
\indent See the function \texttt{insert\_sort} in "main.c".\\

\noindent\textbf{2.} Compose a program to run the same program on two N/2 randomly generated integers and then merge-sort them onto a single output file.\\
\indent My program runs the \texttt{insert\_sort} function on the first half of the array and then on the second half of the array, and then merge them into the 'mergeSort.log'. \\

\noindent\textbf{3.} Verify that the two output files are identical\\
\indent The output for the two sorting approach are "insertSort.log" and "mergeSort.log", we can use the diff command to verify they are the same.\\

\noindent\textbf{4.} Verify the elapsed times of two programs. Explain why the performance differences?\\
\indent The sort with the merge step uses only 50\% of the time of the insertion sort. The complexity of insertion sort for an array of size $N$ is $O(N^2)$, so the complexity of the two insertion sort for an array of size $\frac{N}{2}$ is $\left(\frac{N}{2}\right)^2\times 2=\frac{N^2}{2}$ (the complexity of the merge step is $N$), which is half of the insertion sort complexity.\\
\end{document}